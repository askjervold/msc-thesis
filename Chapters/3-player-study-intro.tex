% Chapter 3

\chapter{Player Study - Introduction}

\label{Chapter3}
\lhead{Chapter 3. \emph{Player Study - Introduction}}

% Section 1 - Survey distribution
\section{Survey Distribution}

The survey was distributed in a Norwegian and an English version, and in multiple phases. After the first phase, some questions were adjusted, and a missing question regarding weight loss and perceived improvement on physical health was added. The final version of the survey can be found in appendix \ref{appendix:survey}.

The first phase of distribution found place at a large Pokémon Go event in \emph{Frognerparken} in Oslo, Norway, a large sculpture park and popular destination for Pokémon Go players due to the high density of pokestops and spawns. The survey was available online via an easily accessible custom URL from a known link shortener, leading to the Norwegian version. Single players and groups of players were approached over the span of about two hours, given a short introduction to the project and asked whether they would be willing to respond to the survey. Given a positive response, they were handed a note with the URL. Extra effort was made to reach out to players in the following categories: Young children (15 or below), parents playing with children, and players above 40. This was done in an attempt to reach the broad spectrum of ages participating, even though the bulk of players are between the ages of 20 and 35. \todo{Should I mention the amount of responses that (likely) originated from here?}

In the second phase, the survey was distributed on \emph{Facebook}. A post explaining the project and an encouragement to respond despite the length of the survey, along with links to both versions, was shared in each of the four largest Norwegian Pokémon Go groups (\emph{Pokémon GO: Norway}, \emph{Pokemon Go Norge}, \emph{Pokémong GO - Oslo \& Akershus} and \emph{Pokémon GO Trondheim}), as well as personally. It was then re-shared by several people to their local Pokémon GO groups or their personal friends and followers. \todo{Could mention an estimate of the amount of people reached through these groups, and the amount of responses it yielded?}

The third phase involved sharing the survey on the popular internet forum \emph{Reddit}. Here it was shared in two \emph{subreddits}, sub-forums on the larger site with their own dedicated communities. The first subreddit was \emph{/r/PokemonGo}, the main community for people interested in the game in general. The second, \emph{/r/TheSilphRoad}, is a community devoted to research on all things related to Pokémon GO.

In addition to the three main phases of distribution, people encountered playing or talking about the game at any time were approached and asked to participate throughout all of September.

In early December, the respondents who had left contact information were contacted with a short follow-up survey, seen in appendix \ref{appendix:followup}. Those who had left comments of note separating them from the other respondents were also asked about these points of interest. This second questionnaire was an attempt to gather information on the longevity of the game, some clarifications of questions from the original survey and a question about spending money in the game. 

\todo{Perhaps include a figure of the distribution of link clicks?}

% Section 2 - Demographics
\section{Demographics}

\todo{A summary of the demographics of the respondents}

The survey collected data on the demographics of its respondents, including the age, gender and the country in which most of their playing occurred. The recorded nation is assumed to be their country of residence in most cases. A total of 2190 responses were recorded.

Out of everyone who responded, 1244 (slightly below 57 \%) were male and the rest, 946 respondents, were female. Their ages are given in the table below.

\todo{Insert table containing the age distribution for male, female and total, perhaps a histogram}

The results contained 1192 responses stating Norway as their main location for playing. Because the survey was distributed in Norwegian Facebook groups, but not in local communities for other countries, Norway has been excluded from the following \todo{table/figure}, which shows the distribution of players per country for the remaining 998 respondents.

\todo{Include table or figure showing the countries of respondents}

The survey also asked for the main occupation of each respondent, with four categories available: employed, unemployed, higher education (e.g. university or college) and lower education (e.g. high school or middle school). The following table again shows the distribution between these categories. \todo{This data might not be very relevant and could possibly be removed.}

\todo{Table of occupations if relevant}

% Section 3 - Mapping to research questions
\section{Survey and Research Questions}

The purpose of the survey was to collect data to answer the research questions presented in section \ref{section:research-questions}. This section will provide a mapping between the questions of the survey and the research questions each of them aim to answer.

The first research goal, examining the success factors of Pokémon GO as seen in section \ref{rg1}, had the research questions seen in column 2, with the corresponding survey questions in column 3 of the following table:

\begin{table}[h!]
	\caption{\emph{What are the main factors that made Pokémon GO successful?} Survey Questions}
	\centering
	\label{tab:rg1-survey-questions}
	\begin{tabular}{|rl|l|}
		\hline
		RQ1.1 & Fisk & Laks\\
		\hline
		RQ1.2 & Torsk & Hyse\\
		\hline
	\end{tabular}
\end{table}