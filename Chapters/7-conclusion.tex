\chapter{Conclusion}
Pokémon GO was a huge success compared to previous similar games, at its peak grossing for more than all other mobile games combined. We studied the possible reasons for its initial success and its subsequent downfall as Pokémon GO players suddenly went from being everywhere to being a rare sight. We also studied the potential effects playing the game had on its players' health, both physical and mental. The study was performed through a survey that garnered more than two thousand responses from players of all ages and activity levels.

Not surprisingly, the results from the survey indicate that the success of Pokémon GO largely comes from the combination of a Pokémon game on smart phones for availability and nostalgia, and a game that encouraged players to go outside and explore during summer when it was warm and they had the time. Players enjoyed following in the footsteps of the protagonists of the franchise, endeavoring to \emph{"catch 'em all"} and \emph{"become the very best"}, and many were excited about trying an augmented reality game, a genre they had not previously tried.

Unfortunately the game was lacking in content and things to do once players had caught all the monsters, and failed to keep most of its players engaged in the long term when they were unable to catch new monsters and school and work were back on their minds and schedules.

The game was very successful at bringing players out to play, when many of them would otherwise have stayed home participating in sedentary activities. Studies show that inactive lifestyles are the cause of many serious health issues, and increasing physical activity through intrinsically motivated activities is an effective way of preventing these kinds of lifestyle diseases. The game was particularly effective in making low activity players more active: where one in every two players were less physically active than the World Health Organization's recommended minimum weekly physical activity before Pokémon GO, only one in five were below the threshold while playing, while the group of players with a high level of activity more than tripled in size.

The increase in physical activity was not lasting, however, with most players falling back to their old ways after they stopped playing. Some players were inspired to start exercise by being exposed to exercise in a fun way, but for most players the increased physical activity was a side effect and not something to strive for without the motivation of Pokémon. Many players were also reckless while being consumed by the game, and risked playing while driving or while walking in traffic, or ventured into dangerous neighborhoods in pursuit of new adventures. Luckily, few serious accidents were caused by their actions.

While the game brought some disappointment and negative behavior, many sufferers of mental conditions such as depression or anxiety found the game helped them in dealing with their conditions through pushing them out of their safe spaces and into the world, where they were exposed to both exercise and human interaction. The strong community feeling of the game also helped foster many new friendships and improved existing relationships of all sorts.

The game showed promise, but ultimately failed to deliver in the long run. The style of game has the potential to have a serious impact on the collective health of society, as long as it manages to keep its players engaged for longer periods, and making a new success may be very possible if one can avoid the pitfalls that Pokémon GO fell into.


\chapter{Future Work}
\todo{Future work in the research, e.g. sample other groups, more on financial impact etc}

\section{Use of Pokémon GO in Serious Contexts}

\todo{Discuss use of Pokémon GO in serious contexts. More focused research on the health effects to possibly use the game as a tool to help patients, and use of the exploration aspect to teach local history. Adding items and rewards to motivate more physical activity?}

\section{Future of Pokémon GO and Similar Games}

\todo{Looking at trends for Pokémon GO, Ingress etc, and more about the reasons people stop playing the games?}
