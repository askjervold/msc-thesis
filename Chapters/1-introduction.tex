% Chapter 1

\chapter{Introduction}

\label{Chapter1}
\lhead{Chapter 1. \emph{Introduction}}

% Section 1 - About Pokémon GO
\section{Pokémon GO}

Pokémon GO is a game for Android and iOS devices developed by Niantic Labs, a former branch of Google, previously known primarily for Ingress, another game for Android and iOS. The game enjoyed massive success during the summer of 2016 with 500 million downloads just in the first two months after its release \todo{[cite interview with John Hanke (Niantic CEO)]}.

The game is played by moving around in the real world, where players encounter Pokémon, digital monsters from one of Nintendo's largest franchises \todo{(do I need a citation for this?)}. These \emph{Pokémon} are then caught, using so-called \emph{Pokéballs}, which can either be purchased for in-game currency (which again can either be earned or purchased with real money) or found at so-called \emph{Pokéstops}, which are scattered across the world at notable landmarks and points of interest.

What the player does with the Pokémon they catch varies with the players' preferences. They can be powered up using resources gained through catching more Pokémon, evolved into different, stronger breeds, and be used to battle other players' Pokémon at \emph{gyms}, which are special places distributed more sparsely than Pokéstops also at points of interest in the real world.

% Section 2 - Motivation
\section{Motivation}

When Pokémon GO released in Norway, it didn't take long for streets and parks to be full of people of all ages playing Pokémon. The people who grew up with Pokémon were out, children were playing with their friends, parents were playing with their children, and even those who had no previous relationship with Pokémon were playing to see what the hype was about. During the day, everywhere you looked, someone was playing. Even during the night you were likely to encounter teenagers and young adults playing.

Strangers who just bumped into each other looking for the same Pokémon were talking to each other, cooperating to take down enemy gyms, sharing good locations for catching certain species and showing off their best Pokémon. People who would otherwise have been sitting at home playing games on their computers or watching TV were instead out walking around trying to find new Pokémon. In the weekends, students were going to parks to play instead of going to night clubs to drink.  Power banks (portable chargers) were sold out everywhere, locals were flocking to tourist attractions, and every online news outlet was full of articles about the game.

It was clear that Pokémon GO was impacting the lives of a large number of people, on a much larger scale than previous games had done. This paper is the result of a project attempting to make sense of not only why Pokémon GO was such a huge success, but also the extent to which the game has impacted the lives of its players.

% Section 3 - Research questions and method
\section{Research Questions and Method}
\label{section:research-questions}

In any evaluation of software it is necessary to maintain metrics by which to evaluate strengths and weaknesses of the software. In line with the GQM method \todo{(write more about the method and include reference)}, to evaluate Pokémon GO, the following three research goals have been defined:

\begin{enumerate}
	\item What are the main factors that made Pokémon GO successful?
	\item What are the effects on physical health from playing Pokémon GO, and how much effort does it take to achieve this effect?
	\item What are the effects on mental health from playing Pokémon GO?
\end{enumerate}

To evaluate these goals, a set of research questions was created for each of them.

% Research Goal 1: Success factors
\subsection{Research Goal 1: Success Factors}
\label{rg1}

The first research goal is \textbf{"What are the main factors that made Pokémon GO successful?"}. This goal has been divided into the following five research questions:

\begin{enumerate}[label=RQ1.{\arabic*}]
	\item What factors most often influenced players to initially start playing Pokémon GO?\label{RQ1.1}
	\item What features of Pokémon GO were used the most?\label{RQ1.2}
	\item What features and aspects of Pokémon GO did players like compared to other similar games (location-based games/mobile games)\label{RQ1.3}
	\item What features and aspects of Pokémon GO did players like compared to previous games in the Pokémon franchise?\label{RQ1.4}
	\item What factors cause players to stop playing Pokémon GO?\label{RQ1.5}
\end{enumerate}

The first research goal aims to identify the factors that made Pokémon GO into the enormous success that it became. By looking at what influenced players to start playing and what features they used and enjoyed, we might be able to identify choices that can be used in the development of new games and applications to achieve similar success. By examining what features were not used, what players disliked and what made players stop playing the game, we could avoid the same pitfalls in future development.

% Research goal 2: Physical health effects
\subsection{Research Goal 2: Physical Health Effects}
\label{rg2}

The second research goal is \textbf{"What are the effects on physical health from playing Pokémon GO, and how much effort does it take to achieve this effect?"}. The following six research questions have been created to answer this:

\begin{enumerate}[label=RQ2.{\arabic*}]
	\item What portion of players increased their weekly physical activity due to playing, and what are the type of game-related activities that lead to increased activity?\label{RQ2.1}
	\item How much did players on average increase their weekly physical activity?\label{RQ2.2}
	\item How much have low-activity players increased their weekly physical activity compared to high-activity players?\label{RQ2.3}
	\item What portion of players have experienced weight loss or other health benefits, perceived or measured, and what benefits do they experience?\label{RQ2.4}
	\item What effect does playing Pokémon GO have on less healthy \todo{(sounds "un-sciency", but unhealthy sounds too strict)} habits, such as drinking or eating fast food?\label{RQ2.5}
	\item What physical risks do players expose themselves and others to due to playing?\label{RQ2.6}
\end{enumerate}

The second research goal looks at the physical health effects of playing Pokémon GO. We want to find how prolonged playing of the game has affected the physical health of players, what activities related to the game give this effect, and how players are motivated to participate in these activities. This can help us evaluate the effectiveness of this type of game as a tool to improve not only the health of single players, but of the population as a whole. We also want to identify ways in which playing the game can negatively affect the health of players or people who are exposed to players \todo{(for reasons?)}.

% Research goal 3: Mental health effects
\subsection{Research Goal 3: Mental Health Effects}
\label{rg3}

The third research goal is \textbf{"What are the effects on mental health from playing Pokémon GO?"}. It has been divided into the following four research questions:

\begin{enumerate}[label=RQ3.{\arabic*}]
	\item What portion of players have increased the time spent socializing with others in their spare time, and what activities contributed to the increase in time?\label{RQ3.1}
	\item How much did players on average increase the weekly time spent socializing with others?\label{RQ3.2}
	\item What portion of players have made new friends or experienced improved existing relationships due to playing Pokémon GO?\label{RQ3.3}
	\item Of players suffering from mental illnesses or other mental conditions, what effects has playing Pokémon GO had on these conditions?\label{RQ3.4}
\end{enumerate}

The third research goal examines the effect participating in a popular trend such as the Pokémon GO phenomenon can have on the mental health of its players. We consider motivation, changes to social dynamics and physical activity, and seek to find the elements that positively affect the players, while identifying the aspects that increase mental health struggles.