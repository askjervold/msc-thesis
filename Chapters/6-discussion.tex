% Chapter 6 - Discussion

\chapter{Discussion}

% Section 1 - General reflection on results
\section{Reflection on Results}

\todo{This section should reflect on the results, and explain any choices or mistakes in the process that may have negatively affected the collection of results. Could also include personal speculation on the answers.}

While the survey managed to bring in large amounts of data, some choices made while constructing the survey may have negatively affected the results.

When asking about the gender of the respondent, there were only two options: male or female. In modern society, people are becoming increasingly aware of how they wish to describe themselves, and many feel they do not fit into the binary division of genders into male and female. The question originally had an option for "Other", but it was revised before the distribution of the survey. The decision to limit the question to two options was made based on the desire to keep the results as clear as possible, and a belief that the number of potential respondents who would choose the third option was too small compared to the other two groups to be statistically significant. However, the choice may have caused some players with valuable feedback and important stories to tell to not respond at all, which may be a bigger cost than what was gained from making the choice.

The majority of the survey responses were collected from members of Pokémon GO Facebook groups and internet forums, who are likely to be more dedicated to and immersed in the game than the average player. The unfortunate consequence of this is that the results gathered don't necessarily reflect the average use of the game. This was attempted counteracted by approaching players "in the wild", but this unfortunately did not yield too many responses.

The survey contained 46 questions, a relatively large number, most of which were mandatory. This may have scared away some number of respondents, and again it is likely that those who decided to answer despite the length of the survey were more dedicated to the game than those who were scared away because of it, further increasing the level of dedication of the average respondent.

A few moves were made to attempt to reduce the impact of the survey being long. First, the questions were split across several pages, grouped together by the theme of the questions. Secondly, questions deemed less important were made optional. Finally, respondents were informed that the survey was long, but that any answers they could provide would be of great help, and that they didn't have to write a lot for each question. It is impossible to know how these choices affected the responses received, and some of them may have worked against their purpose.

After some of the first responses and several comments on the length of the survey and the number of mandatory questions, some more questions were made optional. Not all the questions that were asked were used in the analysis of the results, and in retrospect probably should not have been on the survey at all. Particularly the collection of responses "in the wild" would likely have been more successful with a shorter survey. \todo{Not happy about the way this and the previous two paragraphs were written. Consider rewriting them?}

% Section 2 - Reactions from non-players
\section{Reaction From Non-Players}

\todo{This section should discuss reactions from non-players (general attitudes in the community) and perhaps media.}

% Section 2 - The future of Pokémon Go and similar games
\section{Future of Pokémon Go and Similar Games}

\todo{This section speculates on the future of Pokémon Go and similar games, with basis in observations and the results from the survey.}