% Chapter 6 - Discussion

\chapter{Discussion}
\lhead{Chapter 6. \emph{Discussion}}

% Section 1 - Problems with survey
\section{Problems and Risks of Survey}
\label{sec:problems-with-survey}

While the survey managed to bring in large amounts of data, some choices made while constructing the survey may have negatively affected the results.

\subsection{Collection of Results}

There were several issues with the distribution and some of the questions that may have impacted the number and type of responses collected.

When asking about the gender of the respondent, there were only two options: male or female. In modern society, people are becoming increasingly aware of how they wish to describe themselves, and many feel they do not fit into the binary division of genders into male and female. The question originally had an option for "Other", but it was revised before the distribution of the survey. The decision to limit the question to two options was made based on the desire to keep the results as clear as possible, and a belief that the number of potential respondents who would choose the third option was too small compared to the other two groups to be statistically significant. However, the choice may have caused some players with valuable feedback and important stories to tell to not respond at all, which may be a bigger cost than what was gained from making the choice.

The majority of the survey responses were collected from members of Pokémon GO Facebook groups and internet forums, who are likely to be more dedicated to and immersed in the game than the average player. The unfortunate consequence of this is that the results gathered don't necessarily reflect the average use of the game. This was attempted counteracted by approaching players "in the wild", but this unfortunately did not yield too many responses.

Because the actual URL for the survey was quite long and difficult to enter into a web browser, when players were approached and asked to answer the survey, the link on the note they were handed was shortened using a link shortener. While this made the URL much easier to enter into their browser, some people are skeptical of link shorteners as they do not know where they lead. This may have been a contributing factor to the low number of responses yielded from real world encounters. Sharing the direct link to the survey would most likely not have been more successful, but using Google's own link shortener may have helped. It seems surveys in the field are better performed live as interviews rather than by asking subjects to go to a web address later and answer a long survey.

The survey contained 46 questions, a relatively large number, most of which were mandatory. This may have scared away some number of respondents, and again it is likely that those who decided to answer despite the length of the survey were more dedicated to the game than those who were scared away because of it, further increasing the level of dedication of the average respondent.

A few moves were made to attempt to reduce the impact of the survey being long. First, the questions were split across several pages, grouped together by the theme of the questions. Secondly, questions deemed less important were made optional. Finally, respondents were informed that the survey was long, but that any answers they could provide would be of great help, and that they didn't have to write a lot for each question. It is impossible to know how these choices affected the responses received, and some of them may have worked against their purpose.

After some of the first responses and several comments on the length of the survey and the number of mandatory questions, some more questions were made optional. Not all the questions that were asked were used in the analysis of the results, and in retrospect probably should not have been on the survey at all. Particularly the collection of responses "in the wild" would likely have been more successful with a shorter survey. \todo{Not happy about the way this and the previous two paragraphs were written. Consider rewriting them?}

\subsection{Accuracy of Results}

There were also some issues with the questions that may have affected the accuracy of the collected data.

The questions that presented subjects with several options, followed by an \emph{Other} option where they could elaborate may have lead to the \emph{Other} option to be somewhat underrepresented in the cases where more than one option could be selected. When a fitting option was readily available for selection without requiring a written answer, some respondents may have chosen to only select the simple option and ignoring the option for elaboration. Therefore it is not unlikely that some of these questions, such as the one asking about reasons for starting the game discussed in section \ref{sec:success-factors-initial-interest}, may be lacking some data for some categories.

One question asked whether respondents would keep playing when winter came and the weather got cold. However, as the survey was distributed globally, the question failed to take into account those respondents who lived in warmer climates. For some respondents, the weather during summer was actually too warm to play outside for extended periods of time, and they were looking forward to cooler weather so they could go hunting for Pokémon on the same scale as others had been doing during the summer.

Several of the questions asked about time spent on miscellaneous activities, providing options of ranges such as \emph{2 hours or less}, \emph{4 hours or less}, \emph{8 hours or less} and so on. The goal was to make it easier for respondents to reply with estimates rather than having to figure out the exact amount of time they spend on the activity. A secondary goal was to create groupings of similar answers ahead of time.

This choice had two side-effects, however. The first issue was that the larger options such as \emph{20 hours or less} were valid answers even for those who played only an hour and a half, as 20 hours or less does indeed also include the same amounts that all the smaller options do, even though \emph{2 hours or less} would be the preferred response in this case. During construction of the survey, an alternative version of the options was considered, with options such as \emph{Between 1 and 2 hours}, \emph{Between 2 and 4 hours} and so on. However, it was believed that it was more likely that respondents would be confused by this type of alternative, as a play time of two hours could go in either of these categories, and expected that most respondents would understand to use the smaller alternative if more than one fit their answer. To further increase the odds of subjects choosing the correct option, each of these questions was annotated with \emph{"Please choose the smallest alternative that fits"}.

While it was expected that most respondents would choose the most accurate option, there was another issue with the supplied options. The ranges became wider the longer periods of time they were concerned with. The reason for this was to better capture the differences in low activity groups, and an idea that the difference between 15 and 18 hours of activity was much less significant than the difference between 30 minutes and 2 hours. This resulted in low accuracy for the larger ranges, particularly for the \emph{More than 20 hours} option, which was chosen as a cutoff to avoid too many options. On the question regarding physical activity after Pokémon GO, 67 subjects responded that they were active more than 20 hours per week after they started playing, as opposed to only 11 before. Because of the large number of subjects who increased their activity to this level (a 500 \% increase), it could have been useful to further differentiate ranges within this range. \todo{Might want to check this paragraph for coherency later.}

It is important to note that the question asking about time spent playing per day during peak activity cannot be used as a direct indicator of physical activity per day, as many players also reported that they played while driving or on public transit. This reported time playing should also be taken with a grain of salt when considering the success of the game, as it fails to take into account "inactive" playing, where players have the game open on their phone, but the phone is in their pocket or similar. In these cases one can hardly consider the player to be playing, as the app is simply open to track movement. The follow-up survey asked players about this, and while some said that most of their playing was active, some said as much as 90 \% of their playing was inactive. \todo{This paragraph might not be necessary, at least not in its entirety.}

Several of the questions were worded in a manner where they only looked at possible positive effects, neglecting possible negative effects. One such example was the question about playing during winter, but other questions were worse offenders in this aspect. There were questions asking whether players had become more active, more social or skipped unhealthy activities. These questions didn't really consider those who used to exercise daily but were skipping those workouts to hang out at lures playing Pokémon. People who were already social but whose friends didn't want to play, so they left their friends behind to go play alone. Players who usually would not drink alcohol often, but started going to bars every night because their local bars had the highest concentration of Pokéstops. These people exist, but because the questions were worded to look only at the positive side, they are not properly represented in the results.

The question asking whether the subjects had \emph{"lost weight or otherwise felt more physically healthy"} was also somewhat unfortunately worded. While many, especially in the western world \todo{(citation needed)}, certainly would benefit from losing some weight, there are also those who either don't need to lose weight or in fact are underweight and need to gain weight. The question was optional, and it is very possible that some answers from people in this situation were lost because of these conditions. The value of asking whether the subject had lost weight is limited when we don't know their weight.

Some of those who reported having lost weight also mentioned that they had started exercising more or had started a new diet, not related to Pokémon GO. Others said that they didn't know their weight, but felt like they probably had lost some. These respondents have not been categorized in the \emph{Lost weight} category, as we cannot know whether they have actually lost any weight due to playing the game. \todo{This paragraph could possibly be moved to the results chapter.}

More than one out of every five respondents claimed to be suffering from either depression or some form of anxiety. The WHO and ADAA report that between 5 and 7 \% of the global and American population respectively suffer from these conditions \todo{(insert citation)}. While it is possible that these groups of people are attracted to Pokémon GO, leading to a larger portion of the player base suffering these conditions than in the population on average, there is also a possibility that quite a few of these respondents are undiagnosed and do not actually suffer from these conditions, but think they do. However, even if they do not have the actual diagnosis, if they were experiencing symptoms of these conditions and playing Pokémon GO helped them combat them, the effect is still positive, but we need to keep this in mind when considering the effect on actual mental illnesses.

In addition to the problems with the questions themselves, the categorization of free text answers may have caused some accuracy issues. Particularly the question regarding reasons to quit playing was difficult to categorize, as many players gave multiple reasons as one answer. This resulted in many categories where some of them were quite similar, and most of them having only around 7 \% of the responses. It is quite possible that some of these categories should have been combined. The question could possibly have been split into several questions asking whether certain things had caused the subject to quit or reduce play time, but to avoid leading the answers it was decided to keep it as one question.

% Section 2 - General reflection on results
\section{Reflection on Results}

\todo{This section should reflect on the results, possibly with some personal speculation on the answers.}

Not surprisingly, the results from the survey indicate that the success of Pokémon GO largely comes from the combination of a Pokémon game on smart phones for availability and nostalgia, and a game that encouraged players to go outside and explore during summer.

% Section 3 - Reactions from non-players
\section{Reaction From Non-Players}

\todo{This section should discuss reactions from non-players (general attitudes in the community) and perhaps media.}

% Section 4 - The future of Pokémon Go and similar games
\section{Future of Pokémon Go and Similar Games}

\todo{This section speculates on the future of Pokémon Go and similar games, with basis in observations and the results from the survey.}