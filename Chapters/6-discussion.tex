% Chapter 6 - Discussion

\chapter{Discussion}
\label{chapter:discussion}
\lhead{Chapter \ref{chapter:discussion}. \emph{Discussion}}

% Section 1 - Problems with survey
\section{Problems and Risks of Survey}
\label{sec:problems-with-survey}

While the survey managed to bring in large amounts of data, some choices made while constructing the survey may have affected the results.

\subsection{Issues with Collection of Results}

There were some issues with the distribution and some of the questions that may have impacted the number and type of responses collected.

The majority of the survey responses were collected from members of Pokémon GO Facebook groups and internet forums. Players who are members of these communities are likely to be much more dedicated to and immersed in the game than the average player, who may have just heard about the game from friends or in the media and decided to check out the latest craze and have a few days of fun. This was attempted counteracted by approaching players "in the wild", but this unfortunately did not yield too many responses.

The survey contained 46 questions, a relatively daunting number which is likely to have deterred quite a few potential respondents. Originally, the majority of the questions were mandatory, and it is likely that most of those who decided to answer despite the length of the survey were very dedicated players, while the more casual players found the survey to be too lengthy.

A few counter measures were taken to attempt to reduce the impact of the survey being long. First, the questions were split across several pages, grouped together by the theme of the questions. Secondly, questions deemed less important were made optional. Finally, respondents were informed that the survey was long, but that any answers they could provide would be of great help, and that they did not have to write a lot for each question. It is impossible to know how these choices affected the responses received, and it is possible that one or more of them even have worked against their purpose.

After some of the first responses were delivered, several comments had been made on the length of the survey and the number of mandatory questions. This lead to some more questions being made optional, but the damage may already have been done, as this was several days after the large distribution at the Pokémon GO event discussed in Section \ref{sec:survey-distribution}.

Originally, the plan was to conduct interviews with players in the wild and distribute the written survey to players online. However, as the number of questions on the survey grew large, we considered it unlikely that many players would be willing to participate in interviews as long as those that would be required. Additionally, we wanted to reach out to as many players as possible during the event, and asking them to respond to the survey at a later time would allow more players to be contacted while the park was still crowded and players were open to be approached. Another consideration was that it would take less time away from the players when they were out playing, instead allowing them to respond at their own pace when they had the time.

However, given the low number of responses yielded from approaching these players, it seems likely that constructing a shorter version of the survey and collecting live results from players through interviews using this shorter version may have been more successful. Not all the questions that were asked in the survey were used in the analysis of the results, and in retrospect should not have been on the survey at all. The collection of responses in the wild would certainly have been more successful with a shorter survey, whether it was conducted as an interview or handed out to be responded to at their convenience.

The unfortunate consequence of the issues mentioned above is that the results gathered do not necessarily reflect the average use of the game. The length of the survey and the platform for its distribution are both important factors that are very likely to have affected the selection of players who completed the survey, and the average player who responded to the survey may be quite different indeed from the average player of the game overall. In what areas they differ is difficult to say, but it is not a stretch to assume that the results would indeed have looked a little different had other choices been made.

\subsection{Issues with Accuracy of Results}

There were also some issues with the questions and options that may have affected the accuracy of the collected data.

Many of the questions supplied multiple options the respondents could choose from, and many of these questions allowed multiple options to be selected. When answering these questions, in cases where multiple options apply to the respondent, it is possible that they stopped considering options once they found one that was applicable, despite it being possible to select multiple. One counter measure that was made in an attempt to prevent bias towards the first few options was to randomize the order of the options for each respondent, using the built-in functionality of Google Forms.

However, some questions that presented subjects with several options followed by an \emph{Other} option where they could elaborate. This option always came last, which may have lead to the \emph{Other} option to be underrepresented relative to the other options in the cases where more than one option could be selected. When a fitting option was readily available for selection without requiring a written answer, some respondents may have chosen to only select the simple option and ignoring the option for elaboration. Therefore it is not unlikely that some of these questions, such as the one asking about reasons for starting the game discussed in Sections \ref{sec:success-factors-initial-interest-results} and \ref{sec:success-factors-initial-interest-analysis}, may be lacking some data for some categories.

Several of the questions asked about time spent on miscellaneous activities, providing options of ranges such as \emph{2 hours or less}, \emph{4 hours or less}, \emph{8 hours or less} and so on. The goal was to make it easier for respondents to reply with estimates rather than having to figure out the exact amount of time they spend on the activity. A secondary goal was to create groupings of similar answers ahead of time.

This choice had two side-effects, however. The first issue was that the larger options such as \emph{20 hours or less} were valid answers even for those who played only an hour and a half, as 20 hours or less does indeed also include the same amounts that all the smaller options do, even though \emph{2 hours or less} would be the preferred response in this case. During construction of the survey, an alternative version of the options was considered, with categories such as \emph{Between 1 and 2 hours}, \emph{Between 2 and 4 hours} and so on. However, it was believed that it was more likely that respondents would be confused by this type of alternative, as a play time of two hours could go in either of these categories, and expected that most respondents would understand to use the smaller alternative if more than one fit their answer. To further increase the odds of subjects choosing the correct option, each of these questions was annotated with \emph{"Please choose the smallest alternative that fits"}.

While it was expected that most respondents would choose the most accurate option, there was another issue with the supplied options. The ranges became wider the longer periods of time they were concerned with. The reason for this was to better capture the differences in low activity groups, and an idea that the difference between 15 and 18 hours of activity was much less significant than the difference between 30 minutes and 2 hours. This resulted in low accuracy for the larger ranges, particularly for the \emph{More than 20 hours} option, which was chosen as a cutoff to avoid too many options. On the question regarding physical activity after Pokémon GO, 67 subjects responded that they were active more than 20 hours per week after they started playing, as opposed to only 11 before. Because of the large number of subjects who increased their activity to this level (a 500 \% increase), it could have been useful to further differentiate ranges within this range. Any increase made by players who were already in the highest range prior to Pokémon GO was also not tracked due to this cutoff point.

It is important to note that the question asking about time spent playing per day during peak activity cannot be used as a direct indicator of physical activity per day, as many players also reported that they played while driving or on public transit. This reported time playing should also be taken with a grain of salt when considering the success of the game, as it fails to take into account "inactive" playing, where players have the game open on their phone, but the phone is in their pocket or similar. In these cases one can hardly consider the player to truly be playing, as the app is simply open to track movement. The follow-up survey asked players about this, and while some said that most of their playing was active, some said as much as 90 \% of their playing was inactive.

Several of the questions were worded in a manner where they only looked at possible positive effects, neglecting possible negative effects. Some questions asked whether players had become more active, more social or skipped unhealthy activities. These questions did not really consider those who used to exercise daily but were skipping those workouts to hang out at lures playing Pokémon; people with mental conditions such as anxiety whose conditions were worsened because of their focus on Pokémon GO; people who were already social but whose friends did not want to play, so they left their friends behind to go play alone; players who usually would not drink alcohol often, but started going to bars every night because their local bars had the highest concentration of Pokéstops. These people exist, and some of them mentioned these issues in their comments or open text responses. But because the questions were worded to look only at the positive side, they may not be properly represented in the results, as some of them may have simply chosen to answer \emph{"No"} to those questions.

The question asking whether the subjects had \emph{"lost weight or otherwise felt more physically healthy"} was also somewhat unfortunately worded. While many, especially in the western world \cite{NIDDKoverweight}\cite{WHOobesity}, certainly would benefit from losing some weight, there are also those who either do not need to lose weight or in fact are underweight and need to gain weight. The question was optional, and it is very possible that some answers from people in this situation were lost because of these conditions. The value of asking whether the subject had lost weight is limited when we do not know their weight.

Some of those who reported having lost weight also mentioned that they had started exercising more or had started a new diet, not related to Pokémon GO. Others said that they did not know their weight, but felt like they probably had lost some. These respondents have not been categorized as having lost weight, as we cannot know whether they have actually lost any weight due to playing the game.

More than one out of every five respondents claimed to be suffering from either depression or some form of anxiety. The WHO and ADAA report that between 5 and 7 \% of the global and American population respectively suffer from depression \cite{ADAAdepression}\cite{WHOdepression}, and 18 \% of the American population suffering from anxiety disorders \cite{ADAAanxiety}. While it is possible that these groups of people are attracted to Pokémon GO, leading to a larger portion of the player base suffering these conditions than in the population on average, there is also a possibility that quite a few of these respondents are undiagnosed and do not actually suffer from these conditions, but think they do. However, even if they do not have the actual diagnosis, if they were experiencing symptoms of these conditions and playing Pokémon GO helped them combat them, the effect is still positive, but we need to keep this in mind when considering the effect on actual mental illnesses.

In addition to the problems with the questions themselves, the categorization of open text answers may have caused some accuracy issues. Particularly the question regarding reasons to quit playing was difficult to categorize, as many players gave multiple reasons as one answer. This resulted in many categories where some of them were quite similar, and most of them having only around 7 \% or less of the responses. It is quite possible that some of these categories should have been combined. The question could possibly have been split into several questions asking whether certain things had caused the subject to quit or reduce play time, but to avoid leading the answers it was decided to keep it as one question.

Another similar case is the question asking about mental conditions. Only a few conditions were listed as options, encouraging respondents to list other conditions they were suffering from using the \emph{Other} option. Some of the conditions that were listed had very few sufferers. The small sample sizes for these groups and some of the categories in Chapter \ref{chapter:player-study-success-factors} mean that the statistical significance of the results found for these groups is low. However, while the conclusions drawn from these results may not be entirely accurate, the results still give some indication of potential effects.

\subsection{Minor Issues}

The issues listed in this section may have had an impact on the results gathered, but probably very minor compared to the issues discussed in the previous sections.

When asking about the gender of the respondent, the only two options were male or female. While some potential respondents who did not identify with the binary gender division, causing some interesting responses to be lost before they even started, it is unlikely that the number of responses lost because of this is statistically significant. However, because the gender was not really used in the analysis of the results, the question could have been left out all-together.

Because the actual URL for the survey was quite long and difficult to enter into a web browser, when players were approached and asked to answer the survey, the link on the note they were handed was shortened using a link shortener. This same shortened URL was also shared on Facebook. While this made the URL much easier to enter into their browser, some people are skeptical of link shorteners as they do not know where they lead. This may have been a contributing factor to the low number of responses yielded from real world encounters. Sharing the direct link to the survey would most likely not have been more successful, but using Google's own link shortener may have helped slightly. It is however likely that the effect of this was negligible compared to the length of the survey, which is far more likely to have deterred players from responding.

One question asked respondents whether they would keep playing when winter came and the weather got cold. However, as the survey was distributed globally, the question failed to take into account those respondents who lived in warmer climates. For some respondents, the weather during summer was actually too warm to play outside for extended periods of time, and they were looking forward to cooler weather so they could go hunting for Pokémon on the same scale as others had been doing during the summer. However, as the game largely died down for other reasons before the worst of the fall and winter weather settled in, the importance of this question ended up very low.
