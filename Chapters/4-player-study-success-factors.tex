% Chapter 4 - Player Study - Success factors

\chapter{Player Study - Success Factors}
\lhead{Chapter 4. \emph{Player Study - Success Factors}}

This chapter examines the survey results in the context of determining the success factors of the game. The questions in table \ref{tbl:rg1-survey-questions} and their answers are the main focus of this chapter, but relevant answers to other questions are also included.

\section{Initial Interest}
\label{sec:success-factors-initial-interest}

\todo{About the things that made people start playing and when people started playing based on when the game was released in their country}

\subsection{Results}

To answer research question \ref{RQ1.1}, subjects were asked \emph{"Which of the following factors influenced your decision to start playing Pokémon GO?"}. Table \ref{tbl:initial-interest-options} shows the options that were supplied, as well as the number of respondents for each alternative. Subjects could choose more than one option, and the \emph{Other} choice allowed the respondent to describe other reasons.

\begin{table}[h]
	\caption{\emph{Which of the following factors influenced your decision to start playing Pokémon GO?} options and responses}
	\centering
	\label{tbl:initial-interest-options}
	\begin{tabular}{|l|c|c|}
		\hline
		\textbf{Option} & \textbf{Respondents} & \textbf{\% of total}\\
		\hline\hline
		Nostalgia or previous experience & 1507 & 69.64\\\hline
		Official trailers/promotion & 308 & 14.23\\\hline
		Media coverage & 315 & 14.56\\\hline
		Social media or internet forums & 764 & 35.30\\\hline
		Recommendations from friends/family & 738 & 34.10\\\hline
		The opportunity to get discounts or benefits & 9 & 0.42\\ because of Pokémon Go-related promotions && \\\hline
		Other & 165 & 7.62\\\hline
	\end{tabular}
\end{table}

The answers given for the \emph{Other} option were categorized, and table \ref{tbl:initial-interest-other-categories} shows the categories with multiple answers, with the remaining responses grouped together as \emph{Miscellaneous}.

\begin{table}[h]
	\caption{\emph{Which of the following factors influenced your decision to start playing Pokémon GO?} Other categories}
	\centering
	\label{tbl:initial-interest-other-categories}
	\begin{tabular}{|l|c|c|}
		\hline
		\textbf{Category} & \textbf{Respondents} & \textbf{\% of Other}\\
		\hline\hline
		Pokémon & 43 & 26.06\\\hline
		Ingress & 18 & 10.91\\\hline
		Exercise & 29 & 17.58\\\hline
		Children/family & 26 & 15.76\\\hline
		Social & 7 & 4.24\\\hline
		Technology & 6 & 3.64\\\hline
		Fill outside time & 6 & 3.64\\\hline
		Something to do & 5 & 3.03\\\hline
		Real world & 5 & 3.03\\\hline
		Trends & 4 & 2.42\\\hline
		Gamer & 3 & 1.82\\\hline
		Miscellaneous & 10 & 6.06\\\hline
	\end{tabular}
\end{table}

The \emph{Pokémon} category are respondents who said they started playing simply because it was Pokémon. They consume any product related to the franchise, and won't let a Pokémon game go unplayed. These respondents are primarily around their twenties, and have to an extent grown up with Pokémon.

The \emph{Ingress} category are respondents who had previously played Ingress. Some active Ingress players were part of the Pokémon GO beta because of their participation, while others simply wanted to try another similar game. The \emph{Gamer} category are respondents who identify as gamers and picked up Pokémon GO because it was a new game, despite not having previous experience with either Pokémon or Ingress.

The \emph{Exercise} category are respondents who picked up the game as an exercise app, and wanted to use it as an excuse to walk more or a final push to get out and exercise, while the \emph{Fill outside time} are respondents who were already exercising or walking and wanted something to do during these activities.

The \emph{Children/family} category are respondents who either started playing because they wanted to spend time with their children or other family members who were already playing, or decided together with family members (significant others included) to start playing as a common activity. The \emph{Social} category are those who had similar goals with friends or who merely mentioned they started for the social aspect without specifying who they wished to be social with.

The \emph{Technology} category are respondents who were drawn to the game because of the technology used, be it augmented reality, GPS tracking or otherwise. The \emph{Real world} category are respondents who started playing because of the real world integration. They either wanted to find out what familiar locations had been turned into Pokéstops, or use the game to explore and find new and interesting locations. One subject in this category reported that they started playing because a sculpture they had made had been turned into a Pokéstop.

The \emph{Something to do} category are respondents who picked up the game just as something to do, either because they were bored at vacation or similar, or because they needed a distraction.

The \emph{Trends} category are respondents who started playing to take part in the cultural phenomenon and keep up with current trends.

\todo{Is it ok to start all these paragraphs the same way? \emph{(The X category are respondents ...)}}

\todo{Maybe add something about start time?}

\subsection{Analysis}

As expected, the Pokémon brand played a major role in spreading the game to such a vast number of players. 2164 respondents supplied a reason for downloading the game, and 1507 of them - almost 70 \% - listed \emph{Nostalgia or previous experience} as one of or the only reason they started playing. Releasing a Pokémon game that did not require any additional console (e.g. Nintendo DS) but could be played on the smartphone everyone was already carrying was all but guaranteed to be a success. However, not all games can be Pokémon games, so what other things did Pokémon GO do right that we can use to create other successful games? A little over 30 \% of the respondents did not list nostalgia as a reason, and while some of them may have had memories of watching or playing Pokémon but for some reason or other didn't pick this choice, there were certainly players who had no previous experience or connection with Pokémon.

The opportunity to get discounts or benefits because of Pokémon GO-related promotions did draw a few players, but with less than 0.5 \% of the respondents listing this as a reason, it seems negligible. Those who responded within the \emph{Pokémon} category can be grouped together with the nostalgia responders and should thus be ignored as well.

308 respondents, or just over 14 \%, said they were affected by official trailers or promotional material. This shows that the advertisements \todo{(were there more than the Super Bowl ad?)} were helpful in creating interest in the game. While 14 \% is a relatively small portion compared to the numbers for the other responses, the more casual players who were not reached by the survey (see more in section \ref{sec:problems-with-survey}) may have had a larger portion of players who were intrigued by the Super Bowl ad but who didn't have much previous experience with Pokémon. \todo{Insert picture from the Super Bowl ad here.}

During the first few weeks following the release of the game, there was quite extensive coverage of the game in media all over the world. A little under 15 \% of respondents said that this media coverage affected their decision to start playing Pokémon GO, meaning it was slightly more effective than the official promotional material at building the player base, despite not an insignificant number of the articles posted were negative in their view \todo{(add some citations for negative articles)}. However, the media would not have covered the phenomenon to such a degree had there not already been a huge player base, but if one can succeed in spreading the game to a large enough number of players such that the media starts covering the game, it is not unlikely that one can achieve a similar growth of an additional 15 \% players \todo{(does this argument make sense?)}. Those who responded within the \emph{Trends} category can also be included in this group.

A combined 69 \% of respondents said they started playing because of either recommendations from friends or family or from reading about the game on social media or internet forums. Similar to the \emph{Media coverage} group, these groups required someone else to pick up the game before them, but a huge following is not required for this to have an effect, unlike what is necessary for media coverage to kick in. Thus it could seem that if one can successfully spread ones game to an initial group of players and the game is appealing enough, it can easily spread naturally via them.

Veteran Ingress players were part of this initial group for Pokémon GO, and almost 11 \% of those who listed other reasons were previous Ingress players who either had been granted beta access to Pokémon GO or started playing it because they were previously familiar with the games from the developer. This also indicates that if a developer already has a successful game (even if somewhat niche, like Ingress), it is possible to adjust some parts of the game and release a new one and gain an initial player base from those familiar with your previous games. This is what game developers have been doing for years, and is also in line with \todo{Kiefer et al. (insert citation, "Systematically Exploring Design Space ...)}.

The \emph{Children/family} and \emph{Social} categories are partly related to the idea of creating an initial player base and letting it grow through sharing, but they also highlight the importance of the social aspect of the game. One area where Pokémon GO succeeded is making the game very social, and it becomes more fun to play together with others, as shown by 20 \% of the \emph{Other} responses placing in these categories. Players enjoyed going out to play with their friends and family, having a purpose and something to do while socializing. More on this in section \ref{sub:mental-health-social}. \todo{Add a note and citation for Coulton et al. (Harnessing Player Creativity ...) perhaps?}

The use of augmented reality technology and anchoring to the real world using GPS positioning also succeeded in attracting some number of players. About 7 \% of responses for the \emph{Other} option gave these areas as one of or the sole reason for playing. There still aren't a large number of games using these technologies, and being one of the few that do allowed Pokémon GO to grab a market share by filling a hole. Some examples of other games with relative success in these areas were mentioned in section \ref{sec:prestudy-ar-location-pervasive-games}, but there should still be room for more games in this category, and new, successful games could be created by changing some parameters of the Pokémon GO formula, as described in \todo{Kiefer et al. (ok with another citation of this only two paragraphs after the previous?)}.

Another category related to real world integration is \emph{Fill outside time}. The respondents in this category take advantage of the location-based aspect of the game, where the game progresses simply by moving around. They were looking for an activity to fill the time they were already spending outside, either walking somewhere (e.g. to work, public transit etc.) or exercising, and a game that doesn't require their constant attention and actually progresses based on the activity they were already performing is a better fit than most other mobile games.

Even though Pokémon GO isn't marketed as an exercise app, over 17 \% of respondents who chose the \emph{Other} option said they started playing the game with that exact purpose. The location-based gameplay serves as motivation to get up and out and to move around. With overweight being an increasing problem and sedentary lifestyles becoming increasingly common, we are frequently reminded by health institutions of the importance of physical activity. While most are aware that it is important to be physically active, many struggle with motivation. Not only being able to combine exercise with something fun, but the game actually requiring players to move around to progress makes Pokémon GO the perfect motivation for many players. Other games have also used a similar recipe to success before, as discussed in section \ref{sec:prestudy-ar-location-pervasive-games}. \todo{Is there any point in including this last sentence?}

The \emph{Something to do} and \emph{Gamer} categories can also be more or less ignored, because it is difficult \todo{(impossible?)} to say what will attract these players to your game over another one. The \emph{Gamer}s are likely to pick up any game they stumble upon, and the quality of the game will determine whether or not they will stick with the game. The respondents in the \emph{Something to do} category will similarly start any activity in an attempt to find something that can keep their attention. The best one can do to capture these players is to simply get the game as much exposure as possible to increase the odds of being the first activity or game they happen upon, and make sure the game is good enough to keep their attention. In the \emph{Miscellaneous} category are respondents whose reasons have too small of a sample size to draw any conclusions from.

\section{Important Features}
\label{sec:success-factors-features}

\todo{What features do the players like and use, comparing to other games}

\section{Dwindling Interest}

Despite launching as a soaring success, the Pokémon GO bubble also burst relatively quickly. Player numbers started dwindling near the end of July as seen in figure \ref{fig:player-numbers}, and by mid-August it had lost around 80 \% \todo{(check the number and get citation here)} of its active player base. This section attempts to answer research question \ref{RQ1.5} and determine why Pokémon GO all but faded away so quickly.

\todo{Insert figure of global player numbers}

\subsection{Results}

Table \ref{tbl:still-playing} \todo{(something wrong with the ref!)} shows the distribution of responses for the question \emph{"Are you still playing?"}. Not surprisingly, less than 3 \% of the responses said they had already quit playing, as those who have quit are for the most part no longer following Pokémon GO groups or forums even if they previously were. What is more interesting is that over 40 \% of respondents said they had reduced the amount of time they play compared to their peak. While we don't know how much they reduced their play time, we do know that they were playing less, and there had to be some reason for it.

\begin{table}[h]
	\centering
	\label{tbl:still-playing}
	\begin{tabular}{|l|l|l|}
		\hline
		\textbf{No} & \textbf{Yes} & \textbf{Yes, but less frequently than during my peak}\\
		\hline\hline
		65 & 1245 & 884\\\hline
	\end{tabular}
	\caption{\emph{Are you still playing?} survey question responses}
\end{table}

A total of 610 respondents gave one or more reasons for either quitting the game or reducing the amount of time they spend in the game compared to their peak. These answers were categorized into one or more categories depending on the given reason. Table \ref{tbl:reasons-for-quitting} \todo{(something wrong with the ref!)} shows these categories, along with the number of respondents who gave a reason that was placed in that category, and the portion of the total number of respondents who gave reasons. It should be noted that the input for this question was free text. Because some of the categories are closely related, and because the interpretation of some answers may have been slightly wrong, some respondents may have been placed in one category when they should have been in another similar one instead.

\begin{table}[h]
	\centering
	\label{tbl:reasons-for-quitting}
	\begin{tabular}{|l|c|c|}
		\hline
		\textbf{Category} & \textbf{Respondents} & \textbf{\% of total}\\
		\hline\hline
		Reduced free time & 253 & 41.48\\\hline
		Non-urban & 77 & 12.62\\\hline
		Social aspect & 51 & 8.36\\\hline
		Too repetitive & 45 & 7.38\\\hline
		Lack of features/content & 43 & 7.05\\\hline
		Difficulty of progression & 41 & 6.72\\\hline
		Goal completion & 40 & 6.56\\\hline
		Lack of variation & 40 & 6.56\\\hline
		Hype died down & 39 & 6.39\\\hline
		Removal of tracker & 38 & 6.23\\\hline
		Climate/weather & 28 & 4.59\\\hline
		Community management & 26 & 4.26\\\hline
		Technical requirements & 21 & 3.44\\\hline
		Technical issues & 13 & 2.13\\\hline
		Cheaters/catching up & 13 & 2.13\\\hline
		Lack of purpose/endgame & 9 & 1.48\\\hline
		Lack of incentives & 8 & 1.31\\\hline
		Too all-encompassing & 7 & 1.15\\\hline
	\end{tabular}
	\caption{Reasons for quitting or reducing play time of Pokémon GO}
\end{table}

The largest category by far was \emph{Reduced free time}. \todo{(Maybe move this first sentence to the analysis section)} This category consists of respondents who had either quit playing or reduced the amount of time they play because of reduced free time. The cause of the reduced free time was in most cases related to returning to work or school/studies. The \emph{Climate/weather} category is related, being additional players who decided to play less as summer faded away, but for climate and weather reasons rather than (or in addition to) having less time available.

The \emph{Non-urban} category are respondents who live outside urban centers, either in suburban or rural areas. These areas are not as suited for Pokémon GO playing as urban areas, and is sometimes referred to as \emph{the rural problem}. More on this issue in section \ref{sec:success-factors-quitting-analysis}.

The \emph{Social aspect} category are those who quit or reduced play time because the game fell in popularity with others and they didn't have as many other people to play with. Somewhat related is the \emph{Hype died down} category, which consists of respondents who got bored with the game after the novelty wore off and the game was no longer \emph{"hyped"} as much.

The \emph{Too repetitive}, \emph{Lack of features/content}, \emph{Difficulty of progression}, \emph{Goal completion}, \emph{Lack of variation}, \emph{Lack of purpose/endgame} and \emph{Lack of incentives} categories all have some relation to other categories in this group, but are different enough that they have been kept as separate categories. \todo{This paragraph may not be necessary.}

The \emph{Too repetitive} category focuses on the repetitiveness of the game. These respondents felt like they were doing the same thing over and over and felt their enjoyment of the game (e.g. catching the same Pokémon, evolving them and then transferring them) lessen because of this. The \emph{Lack of variation} category is closely related, consisting of players who were tired of mostly only running into the same Pokémon everywhere. This is also somewhat related to \emph{the rural problem}, as the variety of Pokémon becomes much smaller once you leave the urban areas.

The \emph{Lack of features/content} category is again closely related to the previously mentioned \emph{Too repetitive} category. These players got bored of the game because of the limited number of things one can do within the game itself. You are limited to relatively few activities, as described in section \ref{sec:about-pokemon-go} \todo{(should either add more to this section or write more extensively about it in chapter 2)}, and this was simply not enough to keep the interest of these players in the long run. The \emph{Lack of purpose/endgame} category is relevant here, consisting of players who struggled to find a purpose with the game. There was no clear goal or endgame, and thus nothing for these players to works towards.

The \emph{Difficulty of progression} category respondents were frustrated with how long it takes to make any progress after having played for a while. The related \emph{Lack of incentives} category focuses on the lack of incentives to progress or even play the game.

The \emph{Goal completion} category are players who had certain goals when they started playing. As they reached these goals, they no longer felt any reason to play, or at least no longer played as actively. For most of these players, the goal was to catch one of each Pokémon (or at least those available to them in their region). For others, additional goals were to reach certain player levels in the game.

The \emph{Removal of tracker} category consists of players who were unhappy about the removal of the in-game tool for tracking Pokémon. For the first few weeks after the initial release of the game, players could track Pokémon in the area by using a tool that was available within the game itself. The \emph{tracker} showed from one to three footprints next to an image of the Pokémon, indicating its distance from your current location. As the player got closer, the number of footsteps would decrease. However, around July 18th, this tracker stopped working, never showing less than three footsteps. For some time it was believed that this was caused by a bug, but within two weeks, the footsteps feature was removed entirely. The players in this category were not happy about this, and either reduced play time because of it, or quit playing entirely.

Many players were not happy about the way this issue was handled. The feature was "broken" for a long time without any news from the developers about it, leaving players not knowing whether it was a bug that would be fixed or if it was intentional. For a long time there was a general lack of communication from Niantic regarding the game at all, and the \emph{Community management} category consists of players who were frustrated about this to the point where they no longer wished to play as actively, or in some cases at all.

The \emph{Technical requirements} category are respondents who decided that the technical requirements of the application were too much for them to be willing to play, or who actually were unable to play because of them. One of the requirements that many struggled with was the huge battery drain playing the game caused. The game required the screen to be on, and was constantly using GPS and mobile data, resulting in a battery drain that caused most users to have to charge their phone several times a day. This lead to a massive spike in power bank sales \todo{(maybe add a citation?)}, but for some it was simply not feasible or acceptable to charge multiple times during the day. For others, the issue was with the use of mobile data, having too a too limited amount of data available to let the game use it all. A third, major technical requirement that completely lost a portion of players was the requirement for the device to not be rooted. This was a change made in early September in an attempt to stop cheaters, that also ended up rendering many legitimate users who had rooted their Android devices for one reason or other unable to play the game. Many of these users were infuriated with Niantic, also responding within the \emph{Community management} category.

\todo{Cheaters (spoofers and bots) made it difficult to compete or catch up for new players or if one took a break}

\todo{Technical issues: Bugs and server downtime during peak hours}

\todo{Too all-encompassing}

Other answers that had less than 1 \% representation were not included in the table. Some examples are \todo{...}

\subsection{Analysis}
\label{sec:success-factors-quitting-analysis}

\todo{Release in summer: good plan for getting many players in the short term}

\todo{Tracker very important not only to playability but also the concept and idea of being a Pokémon trainer}

\todo{Possible fixes for the feeling of grinding: more xp/better rewards for catching/evolving better pokes}

\todo{Lack of incentives has been partially fixed with streak bonuses and events}

\todo{Regional Pokémon and releasing Pokémon in waves}

\todo{Technical requirements: Screen on requirement remedied by PoGoPlus and Apple Watch}