% Chapter 4 - Player Study - Success factors

\chapter{Player Study - Success Factors}

This chapter examines the survey results in the context of determining the success factors of the game. The questions in table \ref{tbl:rg1-survey-questions} and their answers are the main focus of this chapter, but relevant answers to other questions are also included.

\section{Initial Interest}

\todo{About the things that made people start playing and when people started playing based on when the game was released in their country}

\subsection{Results}

To answer research question \ref{RQ1.1}, subjects were asked \emph{"Which of the following factors influenced your decision to start playing Pokémon GO?"}. Table \ref{tbl:initial-interest-options} shows the options that were supplied, as well as the number of respondents for each alternative. Subjects could choose more than one option, and the \emph{Other} choice allowed the respondent to describe other reasons.

\begin{table}[h]
	\caption{\emph{Which of the following factors influenced your decision to start playing Pokémon GO?} options and responses}
	\centering
	\label{tbl:initial-interest-options}
	\begin{tabular}{|l|c|}
		\hline
		\textbf{Option} & \textbf{Respondents}\\
		\hline\hline
		Nostalgia or previous experience & 1507\\\hline
		Official trailers/promotion & 308\\\hline
		Media coverage & 315\\\hline
		Social media or internet forums & 764\\\hline
		Recommendations from friends/family & 738\\\hline
		The opportunity to get discounts or benefits & 9 \\ because of Pokémon Go-related promotions & \\\hline
		Other & 165\\\hline
	\end{tabular}
\end{table}

The answers given for the \emph{Other} option were categorized, and table \ref{tbl:initial-interest-other-categories} shows the categories with multiple answers, with the remaining responses grouped together as \emph{Miscellaneous}.

\begin{table}[h]
	\caption{\emph{Which of the following factors influenced your decision to start playing Pokémon GO?} Other categories}
	\centering
	\label{tbl:initial-interest-other-categories}
	\begin{tabular}{|l|c|}
		\hline
		\textbf{Category} & \textbf{Respondents}\\
		\hline\hline
		Pokémon & 43\\\hline
		Ingress & 18\\\hline
		Exercise & 29\\\hline
		Children/family & 26\\\hline
		Social & 7\\\hline
		Technology & 6\\\hline
		Fill outside time & 6\\\hline
		Something to do & 5\\\hline
		Real world & 5\\\hline
		Trends & 4\\\hline
		Gamer & 3\\\hline
		Miscellaneous & 10\\\hline
	\end{tabular}
\end{table}

The \emph{Pokémon} category are respondents who said they started playing simply because it was Pokémon. They consume any product related to the franchise, and won't let a Pokémon game go unplayed. These respondents are primarily around their twenties, and have to an extent grown up with Pokémon.

The \emph{Ingress} category are respondents who had previously played Ingress. Some active Ingress players were part of the Pokémon GO beta because of their participation, while others simply wanted to try another similar game. The \emph{Gamer} category are respondents who identify as gamers and picked up Pokémon GO because it was a new game, despite not having previous experience with either Pokémon or Ingress.

The \emph{Exercise} category are respondents who picked up the game as an exercise app, and wanted to use it as an excuse to walk more or a final push to get out and exercise, while the \emph{Fill outside time} are respondents who were already exercising or walking and wanted something to do during these activities.

The \emph{Children/family} category are respondents who either started playing because they wanted to spend time with their children or other family members who were already playing, or decided together with family members (significant others included) to start playing as a common activity. The \emph{Social} category are those who had similar goals with friends or who merely mentioned they started for the social aspect without specifying who they wished to be social with.

The \emph{Technology} category are respondents who were drawn to the game because of the technology used, be it augmented reality, GPS tracking or otherwise. The \emph{Real world} category are respondents who started playing because of the real world integration. They either wanted to find out what familiar locations had been turned into Pokéstops, or use the game to explore and find new and interesting locations. One subject in this category reported that they started playing because a sculpture they had made had been turned into a Pokéstop.

The \emph{Something to do} category are respondents who picked up the game just as something to do, either because they were bored at vacation or similar, or because they needed a distraction.

The \emph{Trends} category are respondents who started playing to take part in the cultural phenomenon and keep up with current trends.

\todo{Is it ok to start all these paragraphs the same way? \emph{(The X category are respondents ...)}}

\subsection{Analysis}

\todo{While not that many who responded to the survey said they were affected by the official trailers etc, there may still have been quite a few who were intrigued by e.g. the Super Bowl ad, from the more casual community}

\section{Important Features}

\todo{What features do the players like and use, comparing to other games}

\section{Dwindling Interest}

\todo{What made people quit? And when did this happen?}